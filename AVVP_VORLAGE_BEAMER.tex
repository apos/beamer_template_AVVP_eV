% !TEX root = AVVP_VORLAGE_BEAMER.tex
% !TEX program = lualatex


% ------------------------------------------------------------
% Metadata (matches LyX export)
% ------------------------------------------------------------
\title[Kurzversion des Titels für Footer]{Hauptthema bzw. Titel\\
Titel 2. Zeile}
\subtitle{Untertitel, wenn vorhanden}
\author[1. Author, 2. Author]{1.~Author \and 2.~Author}
\institute{\inst{1}Astronomische Vereinigung Vorderpfalz e.V.}
\date{Vereinsabend, 04.02.2026}

\begin{document}

% ------------------------------------------------------------
% Optional background + mode (matches LyX export)
% ------------------------------------------------------------
\AVVPDarkMode
% Note: draft graphicx mode is enabled at the top of this file for speed.
% Background example with credits:
\AVVPBgFillWidthCredits{media/Trouvelot-_The_great_nebula_in_Orion_-_1875.png}{0}{0.20}{trouvelot-orion-1875}

% ------------------------------------------------------------
% Title
% ------------------------------------------------------------
\begin{frame}[plain,noframenumbering]
  \titlepage
\end{frame}

% ------------------------------------------------------------
% Agenda
% ------------------------------------------------------------
\begin{frame}{Agenda}
  \tableofcontents{}
\end{frame}

% ============================================================
% Motivation
% ============================================================
\section{Motivation}

\subsection{Struktur}

\begin{frame}{Strukturierte Vorträge }
Einen Vortrag zu strukturieren ist nicht immer einfach. Hier ein paar
Regeln, die für diese Lösungsvorlage gelten.

\medskip{}

\pause{}
\begin{itemize}
\item Konferenzteilnehmer wissen oft wenig von der Materie des Vortrags.
Deshalb: vereinfachen!
\item In 20 Minuten ist es schon schwer genug, die Hauptbotschaft zu vermitteln.
Deshalb sollten Details ausgelassen werden, selbst wenn dies zu Ungenauigkeiten
oder Halbwahrheiten führt.
\item Falls man Details weglässt, die eigentlich wichtig sind, so bitte
auf der Tonspur. Alle werden damit glücklich sein.
\end{itemize}
\end{frame}

\begin{frame}{Strukturierte Vorträge}
Inhalte sollten in Blöcke gegliedert sein und damit zu einem klar
umrissenen Ziel führen.

\medskip{}

\pause{}
\begin{itemize}
\item Es sollte genau zwei oder drei Abschnitte geben (neben der \% Zusammenfassung).
\item {*}Höchstens{*} drei Unterabschnitte pro Abschnitt.
\item Pro Rahmen sollte man zwischen 30s und 2min reden. Es sollte also
15 bis 30 Rahmen geben.
\end{itemize}
\end{frame}

\subsection{LaTeX und LyX}

\begin{frame}{Warum LaTeX Beamer für Präsentationen?}
\begin{itemize}
\item Klare Trennung von Inhalt und Layout
\item Einheitliches Erscheinungsbild über alle Folien
\item Automatische Struktur durch Abschnitte, Unterabschnitte und Agenda
\item Sehr gut geeignet für Formeln, Abbildungen und Referenzen
\item Stabil und skalierbar auch bei großen Präsentationen
\item Textbasierte Quellen ermöglichen Versionierung und langfristige Reproduzierbarkeit
\end{itemize}
\end{frame}

\begin{frame}{Warum LyX in Kombination mit Beamer?}
\begin{itemize}
\item Grafische Oberfläche für LaTeX ohne manuelles Tippen von Code
\item Strikte Strukturierung: Dokument, Abschnitte und Rahmen
\item Saubere Trennung zwischen Inhaltserstellung und Design
\item Ideal für kollaboratives Arbeiten und spätere Pflege
\item Volle LaTeX-Leistung bleibt jederzeit verfügbar
\end{itemize}
\end{frame}

% ============================================================
% Text strukturieren
% ============================================================
\section{Text strukturieren}

\subsection{Aufzählungen}

\begin{frame}{Itemize und Statements}
\framesubtitle{Untertitel sind optional.}

Folgende Itemize-Regeln sind ganz nützlich:
\begin{itemize}
\item Viel Itemize benutzen.
\item Sehr kurze Sätze oder Satzglieder verwenden.
\item Diese Overlays werden mit dem Pause Stil erzeugt.
\end{itemize}
Und natürlich Aufzählungen:
\begin{enumerate}
\item Now lets numerate
\item and two
\item and three
\end{enumerate}
\end{frame}

\begin{frame}{Konzentration auf einzelne Punkte}
\begin{itemize}
\item <1->Man kann auch Overlay-Spezifikationen benutzen, um Overlays zu
erzeugen.
\item <2->Dies wird als zweites gezeigt.
\item <3->Hiermit können, Punkte in beliebiger Reihenfolge präsentiert
werden
\end{itemize}
\end{frame}

\subsection{Spezielle Umgebungen}

\begin{frame}{Satz / Folgerung}
\begin{theorem}
Auf dem ersten Overlay.
\end{theorem}

\begin{corollary}
Auf dem zweiten Overlay.
\end{corollary}
\end{frame}

\begin{frame}{Blocks}
\begin{block}{Normaler Block}
\begin{itemize}
\item Wird auf allen Overlays angezeigt.
\end{itemize}
\end{block}

\begin{exampleblock}{}
Ein Beispielblocktitel

\begin{itemize}
\item $e^{i\pi}=-1$.
\item $e^{i\pi/2}=i$.
\end{itemize}
\end{exampleblock}
\end{frame}

\begin{frame}{Defintionen}
\begin{definition}
Hier steht eine Definition
\end{definition}

\begin{example}
Und hier ein Beispiel
\end{example}

\begin{proof}
Und hier der Beweis.
\end{proof}
\end{frame}

% ============================================================
% Tabellen
% ============================================================
\section{Tabellen}

\subsection{Alleinstehend}

\begin{frame}{Tabelle mit Überschrift}
\begin{tabular}{|c|c|c|c|}
\hline
Lorem & ipsum & dolor & amet\\
\hline
\hline
orem ipsum &  &  & orem iplor s\\
\hline
 & orem dolor s &  & \\
\hline
 &  & orem dolor s & \\
\hline
\end{tabular}
\end{frame}

% ============================================================
% Bilder
% ============================================================
\section{Bilder}

\subsection{Einzelnes Bild}

\begin{frame}[<{*}>]{Bild}
\includegraphics[width=0.9\columnwidth]{media/EVZL1977.JPG}
\AVVPCreditOverlay{hajnal-haslach-2025}
\end{frame}

\subsection{Gemischte Inhalte}

\begin{frame}[<{*}>]{Bild mit Überschrift }
\begin{columns}[t]
\column[t]{5 cm}
Lorem ipsum dolor sit amet, consetetur sadipscing elitr, sed diam
nonumy eirmod empor invidunt ut labore et dolore magna

aliquyam erat, sed diam voluptua.

\column{1em}

\column[t]{9 cm}
\includegraphics[width=0.9\columnwidth]{media/EVZL1977.JPG}
\AVVPCreditOverlay{hajnal-haslach-2025}
\end{columns}
\end{frame}

\begin{frame}{}
\begin{columns}[t]
\column[t]{4 cm}
\begin{itemize}
\item Rahmen (Folie) mit Bild ohne Überschrift
\item Einzelnes Bild mit Quellennachweis
\item Platz für Bulletpoints
\item ... oder Text
\end{itemize}

\column{1em}

\column[t]{12.5 cm}
\includegraphics[width=0.9\columnwidth]{media/EVZL1977.JPG}
\AVVPCreditOverlay{hajnal-haslach-2025}
\end{columns}
\end{frame}

% ============================================================
% Zusammenfassung
% ============================================================
\section*{Zusammenfassung}

\begin{frame}{Zusammenfassung}
\begin{enumerate}
\item Die erste Hauptbotschaft des Vortrags in ein bis zwei Zeilen.
\item Die zweite Hauptbotschaft des Vortrags in ein bis zwei Zeilen.
\item Eventuell noch eine dritte Botschaft, aber nicht noch mehr.
\end{enumerate}

\vskip0pt plus.5fill
\begin{itemize}
\item Ausblick
  \begin{itemize}
  \item Etwas, was wir noch nicht lösen konnten.
  \item Nochwas, das wir noch nicht lösen konnten.
  \item Uns noch eins
  \end{itemize}
\end{itemize}
\end{frame}

% ============================================================
% Anhang / Bibliographie
% ============================================================
\appendix

\section*{Anhang}
\subsection*{Weiterführende Literatur}

\begin{frame}[t]{Quellenverzeichnis}
\begingroup
\renewcommand*{\bibfont}{\fontsize{8pt}{9pt}\selectfont}\printbibliography
\endgroup
\end{frame}

\end{document}
