% Dieses Dokument beschreibt Aufbau, Struktur und Wirkungsweise der AVVP‑Beamer‑Präambel für unseren Verein.
% Astonomische Vereinigung Vorderpfalz e.V. - https://avvp.de

% LyX preview fix: ensure \Urlmuskip exists even if xurl is loaded early
\providecommand\Urlmuskip{\muskip0}

% ------------------------------------------------------------
% USER CONFIG: external assets (paths)
%   Keep these at the top so users can easily adapt the template.
% ------------------------------------------------------------

% Footer logo (right side in the footline)
\def\AVVPFooterLogoFile{pics/avvp_2019_logo_wortmarke_neg.png}

% Footer logo for LIGHT footline (right side in the footline)
% Use the logo variant intended for a light background.
\def\AVVPFooterLogoLightFile{\AVVPTitleLogoLightFile}

% Title slide logos (auto-selected by Light/Dark mode)
\def\AVVPTitleLogoDarkFile{pics/avvp_2019_logo_wortmarke_neg.png}
\def\AVVPTitleLogoLightFile{pics/avvp_2019_logo_wortmarke_pos_gross.png}

% Creative Commons icon in the footer
\def\AVVPCCIconFile{pics/by-nc-sa.png}

% Title slogan text (shown under the title logo on the title page)
\def\AVVPTitleSloganText{Faszination. Astronomie. Erleben.}

% ------------------------------------------------------------
% USER CONFIG: texts / labels
% ------------------------------------------------------------

% Title of the intermediate agenda (Zwischen‑Agenda) slide
\def\AVVPSubAgendaTitle{Weiter geht's im Text ...}

% Optional: title of the main agenda frame
\def\AVVPMainAgendaTitle{Agenda}

% ------------------------------------------------------------
% USER CONFIG: CI colors (single source of truth)
% ------------------------------------------------------------

\definecolor{AVVPBg}{RGB}{5,23,41}          % #051729
\definecolor{AVVPBlue}{RGB}{46,108,198}     % #2E6CC6
\definecolor{AVVPSpark}{RGB}{235,156,70}    % #EB9C46
\definecolor{AVVPGrey}{RGB}{90,98,110}

% ------------------------------------------------------------
% USER CONFIG: default mode
% ------------------------------------------------------------

% Default presentation mode (set ONE of these)
% NOTE: Do NOT call \AVVPLightMode/\AVVPDarkMode here, because those
% macros are defined later in this file. We only set the default flag.
\newif\ifAVVPDefaultLightMode
% \AVVPDefaultLightModetrue
\AVVPDefaultLightModefalse


% ------------------------------------------------------------
% PACKAGES (safe in Beamer + LyX)
% ------------------------------------------------------------
\usepackage{xcolor}
\usepackage{tikz}
\usetikzlibrary{calc}
\usepackage{amssymb}
\usepackage{etoolbox}
\usepackage{graphicx}

\usepackage{url}

% Media9 for embedding videos/animations (optional)
\usepackage{media9}

% Quotes (ok, uses hyperref indirectly)
\usepackage[autostyle]{csquotes}


% ------------------------------------------------------------
% PDF "Artifact" marking (no tagpdf dependency)
% Purpose: mark purely decorative overlays (e.g. Bildnachweis) so they
% do not become a selectable/tabbable link/annotation target.
% Uses low-level PDF literals that work with LuaTeX.
% ------------------------------------------------------------
\makeatletter
\newcommand{\AVVPArtifactBegin}{%
  \ifdefined\pdfextension
    \pdfextension literal direct { /Artifact BMC }%
  \else
    \pdfliteral direct { /Artifact BMC }%
  \fi
}
\newcommand{\AVVPArtifactEnd}{%
  \ifdefined\pdfextension
    \pdfextension literal direct { EMC }%
  \else
    \pdfliteral direct { EMC }%
  \fi
}
\makeatother

% ------------------------------------------------------------
% ---------- THEME (do not change) ----------
% ------------------------------------------------------------
\usetheme{Malmoe}

% ------------------------------------------------------------
% Ensure shaded nav entries are actually dimmed (mix fg into bg)
% ------------------------------------------------------------
% \setbeamercolor{section in head/foot shaded}{fg=section in head/foot.fg!35!section in head/foot.bg}
% \setbeamercolor{subsection in head/foot shaded}{fg=subsection in head/foot.fg!35!subsection in head/foot.bg}

% ------------------------------------------------------------
% Make section/subsection navigation truly left-aligned:
% remove "hbox to <width>" stretching from Beamer's nav macros
% ------------------------------------------------------------
\usepackage{etoolbox}

\makeatletter
\patchcmd{\insertsectionnavigationhorizontal}{\hbox to #1}{\hbox}{}{\errmessage{Patch failed: insertsectionnavigationhorizontal}}
\patchcmd{\insertsubsectionnavigationhorizontal}{\hbox to #1}{\hbox}{}{\errmessage{Patch failed: insertsubsectionnavigationhorizontal}}
\makeatother

\makeatletter
\let\AVVP@orig@beamer@alert\beamer@alert
\makeatother


% ------------------------------------------------------------
% Nav entries: active gets triangle + normal color
%             inactive gets same hue but dimmed
% ------------------------------------------------------------

% SECTIONS
\setbeamertemplate{section in head/foot}{%
  \usebeamercolor[fg]{section in head/foot}\color{fg}%
  \raggedright
  \raisebox{0.15ex}{\scriptsize$\blacktriangleright$}\hspace{0.4em}%
  \insertsectionhead\hspace*{1.2em}%
}
\setbeamertemplate{section in head/foot shaded}{%
  \usebeamercolor[fg]{section in head/foot}\color{fg!45}%
  \raggedright
  \hspace*{0.5em}\insertsectionhead\hspace*{1.2em}%
}

% SUBSECTIONS
\setbeamertemplate{subsection in head/foot}{%
  \usebeamercolor[fg]{subsection in head/foot}\color{fg}%
  \raggedright
  \raisebox{0.15ex}{\scriptsize$\blacktriangleright$}\hspace{0.4em}%
  \insertsubsectionhead\hspace*{1.0em}%
}
\setbeamertemplate{subsection in head/foot shaded}{%
  \usebeamercolor[fg]{subsection in head/foot}\color{fg!45}%
  \raggedright
  \hspace*{0.5em}\insertsubsectionhead\hspace*{1.0em}%
}

% Dynamic split widths for header: measure left, clamp, use rest
% ------------------------------------------------------------
\newdimen\AVVPHeaderLeft
\newdimen\AVVPHeaderRight
\newsavebox{\AVVPHeaderLeftBox}

\newcommand{\AVVPComputeHeaderWidths}{%
  % Measure natural width of section navigation
  \sbox{\AVVPHeaderLeftBox}{\insertsectionnavigationhorizontal{\paperwidth}{}{}}%
  \AVVPHeaderLeft=\wd\AVVPHeaderLeftBox
  % Clamp left width to max 0.70\paperwidth
  \ifdim\AVVPHeaderLeft>0.70\paperwidth
    \AVVPHeaderLeft=0.70\paperwidth
  \fi
  % Compute right as the remaining width
  \AVVPHeaderRight=\paperwidth
  \advance\AVVPHeaderRight by -\AVVPHeaderLeft
}


% ------------------------------------------------------------
% Header: left-aligned, stable (current section + current subsection)
% ------------------------------------------------------------

\setbeamertemplate{headline}{}


% ------------------------------------------------------------
% Global switch: show agenda bar by default
% ------------------------------------------------------------
\newif\ifAVVPShowAgendaBar
\AVVPShowAgendaBartrue

% Public toggles (global and local)
% - Use \AVVPAgendaBarOff / \AVVPAgendaBarOn in the document preamble to disable/enable globally.
% - Use \AVVPWithAgendaBarOff{...} to disable only for a specific block/frame.
\newcommand{\AVVPAgendaBarOn}{\global\AVVPShowAgendaBartrue}
\newcommand{\AVVPAgendaBarOff}{\global\AVVPShowAgendaBarfalse}
\newcommand{\AVVPWithAgendaBarOn}[1]{\begingroup\AVVPShowAgendaBartrue #1\endgroup}
\newcommand{\AVVPWithAgendaBarOff}[1]{\begingroup\AVVPShowAgendaBarfalse #1\endgroup}

% ------------------------------------------------------------
% Detect ToC usage (any kind) + whether we are inside the Sub-Agenda frame
% ------------------------------------------------------------
\newif\ifAVVPToCThisFrame
\AVVPToCThisFramefalse

\newif\ifAVVPSubAgendaFrame
\AVVPSubAgendaFramefalse

\usepackage{etoolbox}
\makeatletter
\pretocmd{\beamer@tableofcontents}{\global\AVVPToCThisFrametrue}{}%
  {\errmessage{Patch failed: beamer@tableofcontents (pre)}}
\makeatother

% ------------------------------------------------------------
% Helper: render the agenda bar (single place, no duplication)
% ------------------------------------------------------------
\newcommand{\AVVPAgendaBarRender}{%
  \begingroup
  % If the footline is switched to LIGHT, the agenda bar row must also use a light background.
  % Keep the current foreground/text colors unchanged.
  \ifAVVPFootlineLightMode
    \setbeamercolor{section in head/foot}{bg=white}%
    \setbeamercolor{section in head/foot shaded}{bg=white}%
    \setbeamercolor{subsection in head/foot}{bg=white}%
    \setbeamercolor{subsection in head/foot shaded}{bg=white}%
  \fi

  \AVVPComputeHeaderWidths
  \hbox{%
    \begin{beamercolorbox}[
        wd=\AVVPHeaderLeft,
        ht=2.5ex,dp=1.125ex,
        leftskip=1ex,rightskip=0pt
      ]{section in head/foot}
      \insertsectionnavigationhorizontal{\AVVPHeaderLeft}{}{}
    \end{beamercolorbox}%
    \begin{beamercolorbox}[
        wd=\AVVPHeaderRight,
        ht=2.5ex,dp=1.125ex,
        leftskip=0pt,rightskip=1ex
      ]{subsection in head/foot}
      \hfill\insertsubsectionnavigationhorizontal{\AVVPHeaderRight}{}{}
    \end{beamercolorbox}%
  }%
  \endgroup
}

% ------------------------------------------------------------
% Agenda bar (used as top footline) - conditional
% Rule:
% - any ToC frame: hide agenda bar, EXCEPT our Sub-Agenda frame
% - normal frames: show agenda bar
% ------------------------------------------------------------
\newcommand{\AVVPAgendaBar}{%
  \ifAVVPShowAgendaBar
    \ifAVVPToCThisFrame
      \ifAVVPSubAgendaFrame
        \AVVPAgendaBarRender
        \global\AVVPSubAgendaFramefalse
      \else
        % Main agenda / any other ToC frame -> hide agenda bar
      \fi
    \else
      % Normal frame -> show agenda bar
      \AVVPAgendaBarRender
    \fi
  \fi
  \global\AVVPToCThisFramefalse
}



% ------------------------------------------------------------
% Install two-row footline late (after theme overrides)
% ------------------------------------------------------------
\makeatletter
\newcommand{\AVVPInstallTwoRowFootline}{%
  % capture the theme's final footline
  \let\AVVPOrigFootline\beamer@@tmpl@footline
  % replace with two-row footline
  \setbeamertemplate{footline}{%
    \vbox{%
      \offinterlineskip
      \AVVPAgendaBar
      \AVVPOrigFootline
    }%
  }%
}
\makeatother

\AtBeginDocument{\AVVPInstallTwoRowFootline}



% Fonts: make both rows identical
%\setbeamerfont{section in head/foot}{size=\tiny}
%\setbeamerfont{subsection in head/foot}{size=\tiny}
% Agenda uses the same font as the standard Malmoe footer
\setbeamerfont{section in head/foot}{parent=author in head/foot}
\setbeamerfont{subsection in head/foot}{parent=author in head/foot}


% ---------- Colors ----------

% ------------------------------------------------------------
%  Literaturverzeichniseintrag
% ------------------------------------------------------------
% Bibliography: bullet in AVVP orange (instead of the default "document" icon)
\setbeamercolor{bibliography item}{fg=AVVPSpark}
\setbeamertemplate{bibliography item}{\textcolor{AVVPSpark}{\raisebox{0.25ex}{\scriptsize$\blacktriangleright$}}}

% mit cite-key am Anfang + AVVP Bullet (biblatex)
\makeatletter
\AtBeginDocument{%
  \@ifpackageloaded{biblatex}{%

	\newcommand{\AVVPBibBullet}{%
	  {\setbeamercolor{itemize item}{fg=AVVPSpark}\avvpsparkbullet}%
	}%

    \DeclareFieldFormat{entrykey}{[#1]}%
    \renewbibmacro*{begentry}{%
      \AVVPBibBullet\addspace
      \printfield{entrykey}\addspace
    }%

  }{}%
}
\makeatother

% ohne Zeichen zu Beginn
\setbeamertemplate{bibliography item}{}

% Bibliography: URLs in white (AVVP style)
\makeatletter
\AtBeginDocument{%
  \@ifpackageloaded{biblatex}{%
    \DeclareFieldFormat{url}{\textcolor{white}{\url{#1}}}%
  }{}%
}
\makeatother


% ------------------------------------------------------------
% HARD FAIL if a web/online entry has no title
% Rationale: missing title can break biblatex styles/macros and must be fixed in the .bib.
% Applies to entry types: online, webpage
% ------------------------------------------------------------
\makeatletter
\AtBeginDocument{%
  \@ifpackageloaded{biblatex}{%
    \AtDataInput{%
      \ifboolexpr{ test {\ifentrytype{online}} or test {\ifentrytype{webpage}} }{%
        \iffieldundef{title}{%
          \PackageError{AVVP-Beamer}{Bib entry '\thefield{entrykey}' has no 'title' field}{%
            Add a title={...} field to this entry in your .bib file.\MessageBreak
            Example: title={TMS Astro Teil 2}.\MessageBreak
            Compilation is stopped intentionally to prevent broken bibliographies.
          }%
        }{}%
      }{}%
    }%
  }{}%
}
\makeatother


% AVVP Titelseite mit Slogan auf der Titelseite (Light/Dark + Auto-Logo)
\makeatletter
\setbeamertemplate{title page}{%
  \vbox{}
  \vfill
  \begingroup
    \centering

    % --- Title ---
    \setbeamerfont{title}{size=\LARGE}
    {\usebeamerfont{title}\usebeamercolor[fg]{title}\inserttitle\par}

    % --- Subtitle ---
    \ifx\insertsubtitle\@empty\else
      \vskip0.5em
      {\usebeamerfont{subtitle}\usebeamercolor[fg]{subtitle}\insertsubtitle\par}
    \fi

    % --- Date (blue, under subtitle) ---
    \vskip1em
    {\usebeamerfont{date}\color{AVVPBlue}\insertdate\par}

    % --- Authors (blue, under date) ---
    \vskip0.5em
    {\usebeamerfont{author}\color{AVVPBlue}\insertauthor\par}

    % --- Title graphic (logo) ---
    % Prefer an explicitly provided \titlegraphic{...}.
    % If none is provided, use the AVVP default depending on Light/Dark mode.
    \vskip2.2em
    \begingroup
      \setlength{\fboxsep}{0pt}%
      \setlength{\fboxrule}{0pt}%
      \resizebox{0.30\paperwidth}{!}{%
        \ifx\inserttitlegraphic\@empty
          \AVVPTitleLogoAuto
        \else
          \inserttitlegraphic
        \fi
      }\par
    \endgroup

    % --- Slogan (mode-aware, under logo, scaled to same width) ---
    \vskip0.8em
    \resizebox{0.30\paperwidth}{!}{%
      {\AVVPTitleFont\AVVPTitleSloganColor \AVVPTitleSloganText}%
    }\par

    % --- Institute (unchanged) ---
    \vskip2em
    {\usebeamerfont{institute}\insertinstitute\par}

  \endgroup
  \vfill
}
\makeatother

% ------------------------------------------------------------
% Title frame WITHOUT any footer/headline
%   LyX exports often call \makebeamertitle.
%   We override it so the title slide never shows the two-row footline.
% ------------------------------------------------------------
\makeatletter
\renewcommand\makebeamertitle{%
  \begingroup
    % disable any headline/footline for the title frame only
    \setbeamertemplate{headline}{}%
    \setbeamertemplate{footline}{}%
    % keep background (\usebackgroundtemplate) intact
    \begin{frame}[plain,noframenumbering]
      \titlepage
    \end{frame}
  \endgroup
}
\makeatother


% ------------------------------------------------------------
% AVVP Footer: Logo rechts
% ------------------------------------------------------------
% ------------------------------------------------------------
% Footline background mode (independent from Light/Dark mode)
% Default: dark footline (current behavior)
% Usage (document preamble):
%   \AVVPLightMode
%   \AVVPFootlineLight   % optional: make footline background white
% ------------------------------------------------------------
\newif\ifAVVPFootlineLightMode
\AVVPFootlineLightModefalse

\newcommand{\AVVPFootlineLight}{\global\AVVPFootlineLightModetrue}
\newcommand{\AVVPFootlineDark}{\global\AVVPFootlineLightModefalse}

% Footline colors (left + center): always AVVPBg background
\setbeamercolor{author in head/foot}{bg=AVVPBg,fg=white}
\setbeamercolor{title in head/foot}{bg=AVVPBg,fg=white}

\setbeamercolor{avvp foot right}{bg=AVVPBg,fg=white}

% --- AVVP Logo (single source of truth) ---
% Dark footline: current default logo
\pgfdeclareimage[height=3ex]{avvp-logo-dark}{\AVVPFooterLogoFile}
\newcommand{\AVVPLogoDark}{\pgfuseimage{avvp-logo-dark}}

% Light footline: selectable logo (requested)
\pgfdeclareimage[height=3ex]{avvp-logo-light}{\AVVPFooterLogoLightFile}
\newcommand{\AVVPLogoLight}{\pgfuseimage{avvp-logo-light}}

% Auto-select based on the footline background mode
\newcommand{\AVVPLogoAuto}{%
  \ifAVVPFootlineLightMode
    \AVVPLogoLight
  \else
    \AVVPLogoDark
  \fi
}

% ------------------------------------------------------------
% Title slide logo assets (Light/Dark)
% ------------------------------------------------------------
% NOTE: These are used on the title page. The footer uses \AVVPLogo.
\pgfdeclareimage{avvp-title-logo-dark}{\AVVPTitleLogoDarkFile}
\pgfdeclareimage{avvp-title-logo-light}{\AVVPTitleLogoLightFile}

\newcommand{\AVVPTitleLogoDark}{\pgfuseimage{avvp-title-logo-dark}}
\newcommand{\AVVPTitleLogoLight}{\pgfuseimage{avvp-title-logo-light}}

% Mode flag (default: Dark)
\newif\ifAVVPIsLightMode
\AVVPIsLightModefalse

% Helper: auto-select the title logo based on the current mode
\newcommand{\AVVPTitleLogoAuto}{%
  \ifAVVPIsLightMode
    \AVVPTitleLogoLight
  \else
    \AVVPTitleLogoDark
  \fi
}

% Helper: slogan color based on the current mode
\newcommand{\AVVPTitleSloganColor}{%
  \ifAVVPIsLightMode
    \color{AVVPBg}% dark text on light background
  \else
    \color{white}% light text on dark background
  \fi
}

% --- CC BY-NC-SA Icon ---
\pgfdeclareimage[height=1.3ex]{cc-by-nc-sa}{\AVVPCCIconFile}
\newcommand{\AVVPCC}{\pgfuseimage{cc-by-nc-sa}}

% set this, if you do not want to position the logo in the frame, but only in the footer!
\setbeamertemplate{logo}{}

\makeatletter
\setbeamertemplate{footline}{%
  \leavevmode%
  % Optional: force a light (white) footline background
  \ifAVVPFootlineLightMode
    % Left + center must be light background with dark text
    \setbeamercolor{author in head/foot}{bg=white,fg=AVVPBg}%
    \setbeamercolor{title in head/foot}{bg=white,fg=AVVPBg}%
    % Right stays light background; logo switches via \AVVPLogoAuto
    \setbeamercolor{avvp foot right}{bg=white,fg=AVVPBg}%
    % Agenda bar row (top row): make backgrounds light as well
    % Keep current foreground (text) colors unchanged for now.
    \setbeamercolor{section in head/foot}{bg=white}%
    \setbeamercolor{section in head/foot shaded}{bg=white}%
    \setbeamercolor{subsection in head/foot}{bg=white}%
    \setbeamercolor{subsection in head/foot shaded}{bg=white}%
  \else
    % Keep the theme/default colors for the dark footline
  \fi
  \hbox{%
    % --- left: author ---
	\begin{beamercolorbox}[wd=.33\paperwidth,ht=2.5ex,dp=1ex,left]{author in head/foot}%
	  \usebeamerfont{author in head/foot}%
	  \hspace*{1em}%
	  \insertshortauthor%
	  \hspace{0.6em}{\color{AVVPGrey}\textbullet}\hspace{0.6em}%
	  \insertshortdate
	\end{beamercolorbox}%

    % --- center: title ---
    \begin{beamercolorbox}[wd=.34\paperwidth,ht=2.5ex,dp=1ex,center]{title in head/foot}%
      \usebeamerfont{title in head/foot}\insertshorttitle
    \end{beamercolorbox}%

    % --- right: AVVP Logo + CC License ---
    \begin{beamercolorbox}[wd=.33\paperwidth,ht=2.5ex,dp=1ex,right]{avvp foot right}%
      \raisebox{-0.5ex}{\AVVPCC}%
	  \hspace{1em}%
      \raisebox{-0.6ex}{\AVVPLogoAuto}%
      \hspace*{1em}%
    \end{beamercolorbox}%
  }%
}
\makeatother


% -------------------------------------------------------------------
% AVVP SCHRIFTARTEN UND FARBEN
% -------------------------------------------------------------------

% ---------- Fonts (LuaLaTeX / XeLaTeX) ----------
\usepackage{fontspec}
\usefonttheme{professionalfonts}

% ------------------------------------------------
% Base font: Exo 2 (aus lokalen OTF-Dateien)
% ------------------------------------------------
\setsansfont[
  Path = fonts/exo-2/,
  Extension = .otf,
  UprightFont = *-Regular,
  ItalicFont  = *-Italic,
  BoldFont    = *-Bold,
  BoldItalicFont = *-BoldItalic,
  Ligatures = TeX
]{Exo2}

\setmainfont[
  Path = fonts/exo-2/,
  Extension = .otf,
  UprightFont = *-Regular,
  ItalicFont  = *-Italic,
  BoldFont    = *-Bold,
  BoldItalicFont = *-BoldItalic,
  Ligatures = TeX
]{Exo2}

% ------------------------------------------------
% Explizite Bold-Family (für alerted text etc.)
% ------------------------------------------------
\newfontfamily\ExoTwoBold[
  Path = fonts/exo-2/,
  Extension = .otf,
  UprightFont = *-Bold,
  ItalicFont  = *-BoldItalic,
  Ligatures = TeX
]{Exo2}

% ------------------------------------------------
% Title / Section font: Share Tech Mono
% ------------------------------------------------
\newfontfamily\AVVPTitleFont[
  Path = fonts/share_techmono/,
  Extension = .otf,
  UprightFont = Share-TechMono,
  Ligatures = {TeX,NoCommon}
]{Share Tech Mono}

% ------------------------------------------------
% Apply Beamer font roles
% ------------------------------------------------
\setbeamerfont{title}{family=\AVVPTitleFont}
\setbeamerfont{frametitle}{family=\AVVPTitleFont}
\setbeamerfont{section title}{family=\AVVPTitleFont}
\setbeamerfont{subsection title}{family=\AVVPTitleFont}
\setbeamerfont{subtitle}{family=\AVVPTitleFont}
% Blocks: do NOT force bold titles (match "normal" look)

\setbeamerfont{block title}{series=\mdseries}
\setbeamerfont{block title example}{series=\mdseries}

% ------------------------------------------------------------
% Blocks: neutral (didactic), mode-independent
%   Keep Beamer-like grey/green look for readability in BOTH modes.
% ------------------------------------------------------------
\setbeamercolor{block title}{fg=AVVPBg,bg=black!10}
\setbeamercolor{block body}{fg=AVVPBg,bg=black!5}
\setbeamercolor{block title example}{fg=white,bg=green!50!black}
\setbeamercolor{block body example}{fg=AVVPBg,bg=green!6!white}

% ------------------------------------------------------------
% Blocks (Dark Mode fix): ensure list text inside blocks is dark.
% Reason: In Dark Mode we set itemize/enumerate body to white globally,
% which is correct for normal frames but wrong inside light blocks.
% We temporarily override list text colors inside block/exampleblock.
% Also override alerted text color while inside blocks for contrast.
% ------------------------------------------------------------
\AtBeginEnvironment{block}{%
  % Inside light blocks the global (dark-mode) text colors are wrong.
  % Force readable list text and a higher-contrast alert color.
  \setbeamercolor{itemize/enumerate body}{fg=AVVPBg}%
  \setbeamercolor{itemize/enumerate subbody}{fg=AVVPBg}%
  \setbeamercolor{itemize/enumerate subsubbody}{fg=AVVPBg}%
  \setbeamercolor{alerted text}{fg=AVVPSpark!80!black}%
  \setbeamerfont{alerted text}{series=\mdseries,family=\normalfont}%
}

\AtEndEnvironment{block}{%
  % Restore global list + alert colors according to the active mode
  \ifAVVPIsLightMode
    \setbeamercolor{itemize/enumerate body}{fg=AVVPBg}%
    \setbeamercolor{itemize/enumerate subbody}{fg=AVVPBg}%
    \setbeamercolor{itemize/enumerate subsubbody}{fg=AVVPBg}%
    \setbeamercolor{alerted text}{fg=AVVPSpark!80!black}%
    \setbeamerfont{alerted text}{series=\mdseries,family=\normalfont}%
  \else
    \setbeamercolor{itemize/enumerate body}{fg=white}%
    \setbeamercolor{itemize/enumerate subbody}{fg=white}%
    \setbeamercolor{itemize/enumerate subsubbody}{fg=white}%
    \setbeamercolor{alerted text}{fg=AVVPSpark}%
    \setbeamerfont{alerted text}{series=\mdseries,family=\normalfont}%
  \fi
}

\AtBeginEnvironment{exampleblock}{%
  % Example blocks have a light-green background; use a darker alert color
  % so math stays readable when highlighted.
  \setbeamercolor{itemize/enumerate body}{fg=AVVPBg}%
  \setbeamercolor{itemize/enumerate subbody}{fg=AVVPBg}%
  \setbeamercolor{itemize/enumerate subsubbody}{fg=AVVPBg}%
  \setbeamercolor{alerted text}{fg=AVVPSpark!80!black}%
  \setbeamerfont{alerted text}{series=\mdseries,family=\normalfont}%
}

\AtEndEnvironment{exampleblock}{%
  % Restore global list + alert colors according to the active mode
  \ifAVVPIsLightMode
    \setbeamercolor{itemize/enumerate body}{fg=AVVPBg}%
    \setbeamercolor{itemize/enumerate subbody}{fg=AVVPBg}%
    \setbeamercolor{itemize/enumerate subsubbody}{fg=AVVPBg}%
    \setbeamercolor{alerted text}{fg=AVVPSpark!80!black}%
    \setbeamerfont{alerted text}{series=\mdseries,family=\normalfont}%
  \else
    \setbeamercolor{itemize/enumerate body}{fg=white}%
    \setbeamercolor{itemize/enumerate subbody}{fg=white}%
    \setbeamercolor{itemize/enumerate subsubbody}{fg=white}%
    \setbeamercolor{alerted text}{fg=AVVPSpark}%
    \setbeamerfont{alerted text}{series=\mdseries,family=\normalfont}%
  \fi
}

% Ensure Beamer uses sans fonts everywhere
\renewcommand{\familydefault}{\sfdefault}

% Global canvas/text colors (Dark default)
\setbeamercolor{background canvas}{bg=AVVPBg}
\setbeamercolor{normal text}{fg=white,bg=AVVPBg}

% Titles / structure
\setbeamercolor{title}{fg=AVVPBlue}
\setbeamercolor{frametitle}{fg=AVVPBlue}
\setbeamercolor{section title}{fg=AVVPBlue}
\setbeamercolor{subsection title}{fg=AVVPBlue}
\setbeamercolor{subtitle}{fg=AVVPSpark}
\setbeamercolor{structure}{fg=AVVPBlue}

% Author color (default)
\setbeamercolor{author}{fg=AVVPBlue}

% --- Title page: author/institute/date in AVVPBlue ---
\setbeamercolor{author}{fg=AVVPBlue}
\setbeamercolor{institute}{fg=AVVPBlue}
\setbeamercolor{date}{fg=AVVPBlue}

% optional: wenn du "title page" Elemente getrennt steuern willst
\setbeamercolor{title}{fg=AVVPBlue}
\setbeamercolor{subtitle}{fg=AVVPSpark}

% Credit text color (single source of truth)
% Used by both background credits (vertical) and overlay credits (horizontal)
\newcommand{\AVVPCreditColor}{\color{AVVPBlue!70}}

% -------------------------------------------------------------------
% AVVP BULLETPOINTS
%   Mode-aware: color is controlled via beamercolor "AVVPbullet"
%   IMPORTANT: the bullet macros MUST be robust, otherwise Beamer/LyX
%             overlays may drop them and you get fallback diamonds.
% -------------------------------------------------------------------

% No "pause" used! Bullets are highlightet automatically
\beamerdefaultoverlayspecification{<+-|alert@+>}

% Bullet color "variable" (default = Dark look)
\setbeamercolor{AVVPbullet}{fg=AVVPSpark}

% Robust sparkle bullets (TikZ is fragile unless wrapped robustly)
\DeclareRobustCommand{\avvpsparkbullet}{%
  \tikz[baseline=-0.6ex,scale=0.13]{%
    \usebeamercolor[fg]{itemize item}%
    \path[fill=fg,draw=none]
      (0,1)
      .. controls (0.38,0.38) and (0.38,0.38) .. (1,0)
      .. controls (0.38,-0.38) and (0.38,-0.38) .. (0,-1)
      .. controls (-0.38,-0.38) and (-0.38,-0.38) .. (-1,0)
      .. controls (-0.38,0.38) and (-0.38,0.38) .. (0,1)
      -- cycle;%
  }%
}

\DeclareRobustCommand{\avvpsparkbulletsub}{%
  \tikz[baseline=-0.6ex,scale=0.10]{%
    \usebeamercolor[fg]{itemize subitem}%
    \path[fill=fg,draw=none]
      (0,1)
      .. controls (0.38,0.38) and (0.38,0.38) .. (1,0)
      .. controls (0.38,-0.38) and (0.38,-0.38) .. (0,-1)
      .. controls (-0.38,-0.38) and (-0.38,-0.38) .. (-1,0)
      .. controls (-0.38,0.38) and (-0.38,0.38) .. (0,1)
      -- cycle;%
  }%
}

% Bind to Beamer templates (prevents default diamonds)
\setbeamertemplate{itemize item}{\avvpsparkbullet}
\setbeamertemplate{itemize subitem}{\avvpsparkbulletsub}
\setbeamertemplate{itemize subsubitem}{\avvpsparkbulletsub}

% Force sparkle bullets at the start of every itemize environment
\AtBeginEnvironment{itemize}{%
  \setbeamertemplate{itemize item}{\avvpsparkbullet}%
  \setbeamertemplate{itemize subitem}{\avvpsparkbulletsub}%
  \setbeamertemplate{itemize subsubitem}{\avvpsparkbulletsub}%
}


% -------------------------------------------------------------------
% AVVP NUMMERIERUNG
% -------------------------------------------------------------------
\setbeamerfont{enumerate item}{series=\bfseries}


% ------------------------------------------------------------
% AVVP Background (simple + LyX-safe)
%   - fills page width (no distortion)
%   - image is centered; top/bottom may be cropped by clip
% ------------------------------------------------------------

% Apply a slide background:
%   #1 = image file
%   #2 = white overlay opacity (0..1)
%   #3 = black overlay opacity (0..1)
\newcommand{\AVVPBgFillWidth}[3]{%
  \usebackgroundtemplate{%
    \begin{tikzpicture}[remember picture,overlay]
      % clip to the page
      \path[clip] (current page.south west) rectangle (current page.north east);

      % fill WIDTH (no distortion)
      \node[inner sep=0pt] at (current page.center) {%
        \includegraphics[width=1.18\paperwidth]{#1}%
      };

      % overlays (optional)
      \fill[white,opacity=#2] (current page.south west) rectangle (current page.north east);
      \fill[black,opacity=#3] (current page.south west) rectangle (current page.north east);
    \end{tikzpicture}%
  }%
}

% Reset background to plain color
\newcommand{\AVVPBgReset}{\usebackgroundtemplate{}}

% Optional: clear background for following frames
\newcommand{\AVVPBgClear}{\usebackgroundtemplate{}}



% ============================================================
% IMAGE CREDITS via biblatex (single command does it all)
% Preconditions (LyX):
%   - Document Settings -> Bibliography: Stilformat = Biblatex
%   - Prozessor = biber
%   - Insert -> Verzeichnis -> Bib(la)TeX-Literaturverzeichnis...:
%       bib/images.bib als Datenbank hinzufuegen
% ============================================================
\DeclareRobustCommand{\AVVPCiteTitle}[1]{\citefield{#1}{title}}

\makeatletter
\newcommand{\AVVPBgFillWidthCredits}[4]{%
  \@ifpackageloaded{biblatex}{\nocite{#4}}{}%
  \usebackgroundtemplate{%
    \begin{tikzpicture}[remember picture,overlay]

      \path[clip]
        (current page.south west)
        rectangle
        (current page.north east);

      \node[inner sep=0pt]
        at (current page.center)
        {%
          \includegraphics[width=1.18\paperwidth]{#1}%
        };

      \fill[white,opacity=#2]
        (current page.south west)
        rectangle
        (current page.north east);

      \fill[black,opacity=#3]
        (current page.south west)
        rectangle
        (current page.north east);

      \node[
        anchor=south east,
        inner sep=2pt
      ]
        at ([xshift=-2pt,yshift=24pt]current page.south east)
        {%
          \rotatebox{90}{%
            \fontsize{5pt}{6pt}\selectfont
            \AVVPCreditColor%
            Hintergrundbild: [#4]%
          }%
        };

    \end{tikzpicture}%
  }%
}
\makeatother

% ------------------------------------------------------------
% Overlay-Credit (lower right, horizontal)
%   Usage: \AVVPCreditOverlay{key-a,key-b,key-c}
% Notes:
% - Must be visible from the FIRST overlay.
% - Must NOT break Beamer overlay internals (no raw # tokens inside overlay arguments).
% ------------------------------------------------------------
\makeatletter
\ExplSyntaxOn
\cs_new_protected:Npn \avvp_credit_overlay_text:n #1
  {
    \seq_set_split:Nnn \l_tmpa_seq { , } { #1 }
    Bildnachweis:\space
    \seq_map_inline:Nn \l_tmpa_seq { [##1]\space }
  }
\ExplSyntaxOff

% Wrapper without expl3-named control sequences in its body (LyX/Beamer-safe)
\newcommand{\AVVPCreditOverlayText}[1]{%
  \begingroup
    \ExplSyntaxOn
      \csname avvp_credit_overlay_text:n\endcsname {#1}%
    \ExplSyntaxOff
  \endgroup
}

\newcommand{\AVVPCreditOverlay}[1]{%
  \@ifpackageloaded{biblatex}{\nocite{#1}}{}%
  \begingroup
    % Expand the key list once so the overlay argument contains no raw '#'
    \edef\AVVP@CreditKeys{#1}%
    % IMPORTANT:
    % LyX can wrap whole paragraphs in overlay specs (e.g. <+->), which would
    % add extra overlay steps even for non-interactive elements.
    % Counteract this for the credit overlay so it never requires extra clicks.
    \ifcsname c@beamerpauses\endcsname
      \addtocounter{beamerpauses}{-1}%
    \fi
    \begin{tikzpicture}[remember picture,overlay]
      \node[
        anchor=south east,
        inner sep=2pt
      ] at ([xshift=-10pt,yshift=24pt]current page.south east) {%
        \AVVPArtifactBegin
        \fontsize{5pt}{6pt}\selectfont
        \AVVPCreditColor%
        \AVVPCreditOverlayText{\AVVP@CreditKeys}%
        \AVVPArtifactEnd
      };%
    \end{tikzpicture}%
  \endgroup
}
\makeatother


% ------------------------------------------------------------
% AVVP Dark/Light mode (colors + FORCE sparkle itemize templates)
%   Reason: some Beamer theme/template paths reset itemize symbols.
%   This guarantees sparkle bullets in both modes.
% ------------------------------------------------------------

\newcommand{\AVVPDarkMode}{%
  \global\AVVPIsLightModefalse

  \setbeamercolor{background canvas}{bg=AVVPBg}%
  \setbeamercolor{normal text}{fg=white,bg=AVVPBg}%
  \color{white}%

  % Dark mode: list TEXT color (fixes "white" first bullet text)
  \setbeamercolor{itemize/enumerate body}{fg=white}%
  \setbeamercolor{itemize/enumerate subbody}{fg=white}%
  \setbeamercolor{itemize/enumerate subsubbody}{fg=white}%

  % TOC defaults in Dark
  \setbeamercolor{section in toc}{fg=white}%
  \setbeamercolor{subsection in toc}{fg=white}%
  \setbeamercolor{subsection in toc shaded}{fg=white}%

  % Bullet + enumerate colors in Dark
  \setbeamercolor{AVVPbullet}{fg=AVVPSpark}%
  \setbeamercolor{itemize item}{fg=AVVPSpark}%
  \setbeamercolor{itemize subitem}{fg=AVVPSpark}%
  \setbeamercolor{itemize subsubitem}{fg=AVVPSpark}%

  \setbeamercolor{enumerate item}{fg=AVVPSpark}%
  \setbeamercolor{enumerate subitem}{fg=AVVPSpark}%
  \setbeamercolor{enumerate subsubitem}{fg=AVVPSpark}%

  % FORCE sparkle bullets (prevents default diamonds)
  \setbeamertemplate{itemize item}{\avvpsparkbullet}%
  \setbeamertemplate{itemize subitem}{\avvpsparkbulletsub}%
  \setbeamertemplate{itemize subsubitem}{\avvpsparkbulletsub}%

  % Active list text (current overlay item)
  \setbeamercolor{alerted text}{fg=AVVPSpark}%
  \setbeamerfont{alerted text}{series=\mdseries,family=\normalfont}%

  % Blocks: neutral (didactic), mode-independent
  \setbeamercolor{block title}{fg=AVVPBg,bg=black!10}%
  \setbeamercolor{block body}{fg=AVVPBg,bg=black!5}%
  \setbeamercolor{block title example}{fg=white,bg=green!50!black}%
  \setbeamercolor{block body example}{fg=AVVPBg,bg=green!6!white}%

  \setbeamercovered{transparent=45}

}

\newcommand{\AVVPLightMode}{%

  \global\AVVPIsLightModetrue

  \setbeamercolor{background canvas}{bg=white}%
  \setbeamercolor{normal text}{fg=AVVPBg,bg=white}%
  \color{AVVPBg}%

  % Covered (not-yet-shown) text should be lighter in Light Mode
  \setbeamercolor{covered text}{fg=AVVPBg!55}%

  % Light mode: list TEXT color (fixes "white" first bullet text)
  \setbeamercolor{itemize/enumerate body}{fg=AVVPBg}%
  \setbeamercolor{itemize/enumerate subbody}{fg=AVVPBg}%
  \setbeamercolor{itemize/enumerate subsubbody}{fg=AVVPBg}%

  % TOC defaults in Light
  \setbeamercolor{section in toc}{fg=AVVPBg}%
  \setbeamercolor{subsection in toc}{fg=AVVPBg}%
  \setbeamercolor{subsection in toc shaded}{fg=AVVPBg}%

  % Active bullets/numbers = AVVPSpark
  \setbeamercolor{AVVPbullet}{fg=AVVPSpark}%
  \setbeamercolor{itemize item}{fg=AVVPSpark}%
  \setbeamercolor{itemize subitem}{fg=AVVPSpark}%
  \setbeamercolor{itemize subsubitem}{fg=AVVPSpark}%

  % FORCE sparkle bullets (prevents default diamonds)
  \setbeamertemplate{itemize item}{\avvpsparkbullet}%
  \setbeamertemplate{itemize subitem}{\avvpsparkbulletsub}%
  \setbeamertemplate{itemize subsubitem}{\avvpsparkbulletsub}%

  \setbeamercolor{enumerate item}{fg=AVVPSpark}%
  \setbeamercolor{enumerate subitem}{fg=AVVPSpark}%
  \setbeamercolor{enumerate subsubitem}{fg=AVVPSpark}%

  % Active list text (current overlay item)
  \setbeamercolor{alerted text}{fg=AVVPSpark!80!black}%
  \setbeamerfont{alerted text}{series=\mdseries,family=\normalfont}%

  % Blocks: neutral (didactic), mode-independent
  \setbeamercolor{block title}{fg=AVVPBg,bg=black!10}%
  \setbeamercolor{block body}{fg=AVVPBg,bg=black!5}%
  \setbeamercolor{block title example}{fg=white,bg=green!50!black}%
  \setbeamercolor{block body example}{fg=AVVPBg,bg=green!6!white}%

  \setbeamercovered{transparent=60}

}

% ============================================================
% AVVP HEADLINE + TOC (Agenda)  --- SINGLE SOURCE OF TRUTH ---
%   - Malmoe headline colors only (layout untouched)
%   - Arrow in subsection headline (LyX-safe)
%   - Agenda TOC: section blue, subsections white, current orange+arrow
%   - Auto Agenda frame at each subsection start (NO ERT needed)
% ============================================================

% ------------------------------
% Headline colors (Malmoe)
% ------------------------------
\setbeamercolor{section in head/foot}{bg=black, fg=AVVPBlue}
\setbeamercolor{section in head/foot shaded}{bg=black, fg=white!60}

\setbeamercolor{subsection in head/foot}{bg=AVVPBg, fg=AVVPSpark}
\setbeamercolor{subsection in head/foot shaded}{bg=AVVPBg, fg=white!60}

% ------------------------------
% Arrow in subsection headline (define ONCE)
%   Uses \blacktriangleright (requires amssymb -> already loaded above)
% ------------------------------
\makeatletter
\let\AVVP@orig@insertsubsectionhead\insertsubsectionhead
\renewcommand{\insertsubsectionhead}{%
  \ensuremath{\blacktriangleright}\,\AVVP@orig@insertsubsectionhead%
}
\makeatother

% ------------------------------
% Agenda / TOC highlighting (current subsection)
%   Beamer uses:
%     - "subsection in toc"        = CURRENT subsection
%     - "subsection in toc shaded" = NON-current subsections
%   So: put ORANGE style into "subsection in toc"
% ------------------------------
\makeatletter
\providecommand{\AVVPtocCurrentSubsection}{}% avoid "already defined"
\renewcommand{\AVVPtocCurrentSubsection}{%
  \begingroup

  % ----- Sections: blue -----
  \setbeamertemplate{section in toc}{%
    {\color{AVVPBlue}\inserttocsection}\par%
  }%
  \setbeamertemplate{section in toc shaded}{%
    {\color{AVVPBlue}\inserttocsection}\par%
  }%

  % ----- Subsections -----
  % CURRENT subsection (orange + arrow)
  \setbeamertemplate{subsection in toc}{%
    {\color{AVVPSpark}\bfseries \ensuremath{\blacktriangleright}\ \inserttocsubsection}\par%
  }%
  % NON-current subsections (mode-aware via beamercolor)
  \setbeamertemplate{subsection in toc shaded}{%
    {\usebeamercolor[fg]{subsection in toc shaded}\color{fg}\inserttocsubsection}\par%
  }%

  % Activates highlighting/shading logic
  \tableofcontents[currentsection,currentsubsection]

  \endgroup
}%
\makeatother

% ============================================================
% GLOBAL TOC STYLE (so standard \tableofcontents matches agenda)
% ============================================================

% Sections: blue
\setbeamertemplate{section in toc}{%
  {\color{AVVPBlue}\inserttocsection}\par%
}
\setbeamertemplate{section in toc shaded}{%
  {\color{AVVPBlue}\inserttocsection}\par%
}

% Unterabschnitte im Inhaltsverzeichnis:
% - Farbe ueber beamercolor (Light: AVVPBg, Dark: white)
\setbeamertemplate{subsection in toc}{%
  {\usebeamercolor[fg]{subsection in toc}\color{fg}\inserttocsubsection}\par%
}
\setbeamertemplate{subsection in toc shaded}{%
  {\usebeamercolor[fg]{subsection in toc shaded}\color{fg}\inserttocsubsection}\par%
}

% ------------------------------
% Agenda frame macro
% ------------------------------
\newcommand{\AVVPAgendaFrame}{%
  {%
    % Mode-aware background + text colors
    \ifAVVPIsLightMode
      % Light mode: white background, dark text
      \usebackgroundtemplate{%
        \begin{tikzpicture}[remember picture,overlay]
          \fill[white] (current page.south west) rectangle (current page.north east);
        \end{tikzpicture}%
      }%
      \setbeamercolor{normal text}{fg=AVVPBg}%
    \else
      % Dark mode: AVVPBg background, white text
      \usebackgroundtemplate{%
        \begin{tikzpicture}[remember picture,overlay]
          \fill[AVVPBg] (current page.south west) rectangle (current page.north east);
        \end{tikzpicture}%
      }%
      \setbeamercolor{normal text}{fg=white}%
    \fi
    \setbeamercolor{frametitle}{fg=AVVPBlue}%

    \global\AVVPSubAgendaFrametrue
    \begin{frame}<beamer>
      \frametitle{\AVVPSubAgendaTitle}
      \AVVPtocCurrentSubsection
    \end{frame}
  }%
}

% ------------------------------
% Switch: where to auto-insert the "Zwischen-Agenda" frame?
% Default: subsections (current behavior)
% Usage in document preamble:
%   \AVVPAgendaOnlySections
%   \AVVPAgendaSectionsAndSubsections
% ------------------------------
\newif\ifAVVPAgendaAtSection
\AVVPAgendaAtSectionfalse

\newif\ifAVVPAgendaAtSubsection
\AVVPAgendaAtSubsectiontrue

\newcommand{\AVVPAgendaOnlySections}{%
  \global\AVVPAgendaAtSectiontrue
  \global\AVVPAgendaAtSubsectionfalse
}

\newcommand{\AVVPAgendaSectionsAndSubsections}{%
  \global\AVVPAgendaAtSectionfalse
  \global\AVVPAgendaAtSubsectiontrue
}

% Section-level agenda content (shows current section overview)
% - Highlight CURRENT section (arrow + bold, like the footer)
% - Keep subsections readable (mode-aware)
\makeatletter
\providecommand{\AVVPtocCurrentSection}{}% avoid "already defined"
\renewcommand{\AVVPtocCurrentSection}{%
  \begingroup

    % ----- Sections -----
    % CURRENT section (bold + arrow)
    \setbeamertemplate{section in toc}{%
      {\color{AVVPBlue}\bfseries \ensuremath{\blacktriangleright}\ \inserttocsection}\par%
    }%
    % NON-current sections
    \setbeamertemplate{section in toc shaded}{%
      {\color{AVVPBlue}\inserttocsection}\par%
    }%

    % ----- Subsections (within the current section) -----
    \setbeamertemplate{subsection in toc}{%
      {\usebeamercolor[fg]{subsection in toc}\color{fg}\inserttocsubsection}\par%
    }%
    \setbeamertemplate{subsection in toc shaded}{%
      {\usebeamercolor[fg]{subsection in toc shaded}\color{fg}\inserttocsubsection}\par%
    }%

    % Render current section overview
    \tableofcontents[currentsection]

  \endgroup
}%
\makeatother

% A dedicated agenda frame for section starts
\newcommand{\AVVPAgendaFrameSection}{%
  {%
    % Mode-aware background + text colors
    \ifAVVPIsLightMode
      % Light mode: white background, dark text
      \usebackgroundtemplate{%
        \begin{tikzpicture}[remember picture,overlay]
          \fill[white] (current page.south west) rectangle (current page.north east);
        \end{tikzpicture}%
      }%
      \setbeamercolor{normal text}{fg=AVVPBg}%
    \else
      % Dark mode: AVVPBg background, white text
      \usebackgroundtemplate{%
        \begin{tikzpicture}[remember picture,overlay]
          \fill[AVVPBg] (current page.south west) rectangle (current page.north east);
        \end{tikzpicture}%
      }%
      \setbeamercolor{normal text}{fg=white}%
    \fi
    \setbeamercolor{frametitle}{fg=AVVPBlue}%

    \global\AVVPSubAgendaFrametrue
    \begin{frame}<beamer>
      \frametitle{\AVVPSubAgendaTitle}
      \AVVPtocCurrentSection
    \end{frame}
  }%
}

% ------------------------------
% AUTO: show Agenda at start of each subsection
% (=> NO ERT needed in the document)
% ------------------------------
\AtBeginSubsection[]{%
  \ifAVVPAgendaAtSubsection
    \begingroup
      \edef\secname{\insertsectionhead}%
      \edef\subsecname{\insertsubsectionhead}%
    \endgroup
    \AVVPAgendaFrame
  \fi
}

% ------------------------------
% AUTO: show Agenda at start of each section (optional)
% ------------------------------
\AtBeginSection[]{%
  \ifAVVPAgendaAtSection
    \AVVPAgendaFrameSection
  \fi
}

% ------------------------------------------------------------
% Apply default mode at begin document (after all macros exist)
% ------------------------------------------------------------
\AtBeginDocument{%
  \ifAVVPDefaultLightMode
    \AVVPLightMode
  \else
    \AVVPDarkMode
  \fi
}


% ------------------------------------------------------------
% VIDEO HELPERS (LyX-friendly)
% - Use a normal image as poster inside the frame.
% - Make the poster clickable:
%     - local file:   file:media/<file>.mp4
%     - URL:          https://...
% Notes:
% - PDF viewers may block "file:" links for security reasons.
% - Autoplay on frame open is not reliably possible across viewers.
% ------------------------------------------------------------
% Poster image for video (no centering here; centering outside if needed)
\newcommand{\AVVPVideoPoster}[2][]{%
  \includegraphics[
    width=\linewidth,
    height=0.70\textheight,
    keepaspectratio,
    #1%
  ]{#2}%
}

% Helper: build a PDF-friendly file URI for local media files
% Notes:
% - Hyperref needs a fully expanded string as the target.
% - Spaces must be percent-encoded (%20) for reliable PDF URI handling.
% - IMPORTANT: expl3 function names contain '_' and must be tokenized with expl3 syntax ON
%   at *definition time* (not only at runtime). Otherwise you get "\tl _set:Nn" and crashes.
% expl3 helper for building file URIs
\makeatletter
\ExplSyntaxOn
\tl_new:N \g_avvp_file_uri_tl

\cs_new_protected:Npn \avvp_build_file_uri:n #1
  {
    % Start with the raw path
    \tl_set:Nn \l_tmpa_tl {#1}
    % Replace literal spaces with %20
    \tl_replace_all:Nnn \l_tmpa_tl {\c_space_tl} {\c_percent_str 20}
    % Decide absolute vs relative: absolute starts with '/'
    \str_if_eq:eeTF {\str_head:n {#1}} { / }
      {\tl_gset:Nx \g_avvp_file_uri_tl { file:///\tl_use:N \l_tmpa_tl }}
      {\tl_gset:Nx \g_avvp_file_uri_tl { file:\tl_use:N \l_tmpa_tl }}
  }

% Public wrapper that defines \AVVP@FileURI as an expanded string for \href
% IMPORTANT: \AVVP@FileURI must survive the local group, because it is used
% afterward in \href{\AVVP@FileURI}{...}.
\newcommand{\AVVPBuildFileURI}[1]{%
  \begingroup
    \avvp_build_file_uri:n {#1}% sets \g_avvp_file_uri_tl
    % Make the result available outside this group:
    \xdef\AVVPFileURI{\tl_use:N \g_avvp_file_uri_tl}% fully expanded
  \endgroup
}
\ExplSyntaxOff
\makeatother

% Helper: pretty label under local media poster
% - Show only "folder/filename" on Unix/mac paths
% - Show only "folder\filename" on Windows paths
% - If there is no folder component, show the original argument
\makeatletter
\ExplSyntaxOn
\cs_new_protected:Npn \avvp_video_local_label:n #1
  {
    % Decide separator based on the presence of a backslash
    \tl_if_in:nnTF {#1} {\c_backslash_str}
      {
        % Windows-style path
        \seq_set_split:Nnn \l_tmpa_seq {\c_backslash_str} {#1}
        \int_compare:nNnTF {\seq_count:N \l_tmpa_seq} > {1}
          {
            \tl_set:Nx \l_tmpb_tl {\seq_item:Nn \l_tmpa_seq {-2}}
            \tl_set:Nx \l_tmpc_tl {\seq_item:Nn \l_tmpa_seq {-1}}
            \texttt{\nolinkurl{\tl_use:N \l_tmpb_tl}\textbackslash\nolinkurl{\tl_use:N \l_tmpc_tl}}
          }
          {\texttt{\nolinkurl{#1}}}
      }
      {
        % Unix/mac-style path
        \seq_set_split:Nnn \l_tmpa_seq {/} {#1}
        \int_compare:nNnTF {\seq_count:N \l_tmpa_seq} > {1}
          {
            \tl_set:Nx \l_tmpb_tl {\seq_item:Nn \l_tmpa_seq {-2}}
            \tl_set:Nx \l_tmpc_tl {\seq_item:Nn \l_tmpa_seq {-1}}
            \texttt{\nolinkurl{\tl_use:N \l_tmpb_tl/\tl_use:N \l_tmpc_tl}}
          }
          {\texttt{\nolinkurl{#1}}}
      }
  }
\newcommand{\AVVPVideoLocalLabel}[1]{\avvp_video_local_label:n {#1}}
\ExplSyntaxOff
\makeatother

% Clickable poster -> open LOCAL media file (external player)
%   #1 optional includegraphics options
%   #2 local target (e.g. media/MyVideo.mp4)
%   #3 poster image (e.g. media/MyVideo_poster.jpg)
\newcommand{\AVVPVideoLocal}[3][]{%
  \begin{center}%
    \AVVPBuildFileURI{#2}% defines \AVVPFileURI (expanded string)
    \href{\AVVPFileURI}{\AVVPVideoPoster[#1]{#3}}\\[0.4ex]%
    {\scriptsize\href{\AVVPFileURI}{\AVVPVideoLocalLabel{#2}}}%
  \end{center}%
}

% Clickable poster -> open URL (e.g. YouTube / Google Drive / website)
%   #1 optional includegraphics options
%   #2 URL
%   #3 poster image
% Clickable poster -> open URL (e.g. YouTube / Google Drive / website)
\newcommand{\AVVPVideoURL}[3][]{%
  \begin{center}%
    \href{#2}{\AVVPVideoPoster[#1]{#3}}%
  \end{center}%
}